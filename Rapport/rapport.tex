\documentclass[a4paper, 10pt, french]{article}

\usepackage[utf8]{inputenc}
\usepackage[T1]{fontenc}
\usepackage[frenchb]{babel}
\usepackage{lmodern}
\usepackage[autolanguage]{numprint}
\usepackage{enumitem}
\usepackage{array}
\usepackage{multirow}
\usepackage{collcell}

\usepackage[margin=3cm]{geometry}
\usepackage{multicol}
\usepackage[10pt]{moresize}
\usepackage{pdflscape}


\usepackage{amsthm}
\usepackage{amsmath}
\usepackage{amssymb}
\usepackage{mathrsfs}
\usepackage{amsopn}
\usepackage{stmaryrd}


\usepackage{listings}
\usepackage{minted}

\usemintedstyle[OCaml]{frame=lines,style=colorful,fontfamily=tt,breaklines}
\renewcommand{\listingscaption}{Listing}
\renewcommand{\listoflistingscaption}{Table des listings}


\newcommand{\netlist}{\emph{net-list}}
\newcommand{\minijazz}{\bsc{MiniJazz}}
\newcommand{\mips}{\emph{MIPS}}
\newcommand{\code}[1]{\texttt{#1}}

\newcommand{\affects}{$\leftarrow$}
\newcolumntype{R}{>{\ttfamily}{l}}
\newcolumntype{M}{>{$}{c}<{$}}
\newcolumntype{U}{>{\code{r1,}}{c}}
\newcolumntype{D}{>{\code{r2,}}{c}}
\newcolumntype{O}{>{\code{o}}{c}}
\newcolumntype{E}{>{$r_1$}{c}}
\newcolumntype{A}{>{\affects}{l}}



\title{Choix relatifs à l'implémentation du processeur}
\author{Marc \bsc{Ducret} \and Florentin \bsc{Guth} \and Martin \bsc{Ruffel} \and Lionel \bsc{Zoubritzky}}

\begin{document}

\maketitle

\section{Structure du système}

\subsection{Architecture du processeur}

On choisit de stocker les mots sur 32 bits (4 octets). En effet, cela permettra d'utiliser des entiers (signés seulement, pour simplifier) de taille importante. On utilise 32 registres, ce qui, en les stockant sur 5 bits, laisse 17 bits pour l'\code{opcode} des instructions à 3 paramètres.

\subsection{Caractéristiques de la \emph{RAM}}

On souhaiterait pouvoir effectuer de l'affichage sur un écran de 256$\times$256 pixels avec des couleurs de la taille d'un mot (32 bits). Cela nécessite donc 256 Ko d'espace mémoire. On choisit donc une \emph{RAM} de 512 Ko.

La \emph{RAM} contiendra des emplacements réservés pour :
\begin{itemize}
  \item initialiser le microprocesseur,
  \item contenir le programme de lancement (la montre digitale),
  \item afficher une \emph{bitmap} à l'écran,
  \item gérer les entrées restantes : changement de mode de la montre, remise à l'heure \ldots
\end{itemize}


\section{Opérations implémentées}

\subsection{Opérations arithmético-logiques}

On conserve pratiquement toutes les opérations de base du \mips.

\begin{table}[H]
  \centering
  \begin{tabular}{|RUDO|EAMMM|}
    \hline
    \multicolumn{4}{|c|}{Opération} & \multicolumn{5}{c|}{Effet} \\
    \hline
    add & & & & & & r_2 &+& o \\
    sub & & & & & & r_2 &-& o \\
    mul & & & & & & r_2 &\times& o \\
    div & & & & & & r_2 &\div& o \\
    sll & & & & & & r_2 &\mathrm{lsl}& o \\
    srl & & & & & & r_2 &\mathrm{lsr}& o \\
    \hline
    and & & & & & & r_2 &\wedge& o \\
    or  & & & & & & r_2 &\vee& o \\
    xor & & & & & & r_2 &\oplus& o \\
    \hline
    slt & & & & & & r_2 &<& o \\
    sle & & & & & & r_2 &\leq& o \\
    seq & & & & & & r_2 &=& o \\
    sne & & & & & & r_2 &\neq& o \\
    \hline
  \end{tabular}
  \caption{Opérations arithmético-logiques}
\end{table}

On ajoute néanmoins les opérations suivantes, afin de faciliter les opérations sur les heures :

\begin{table}[H]
  \centering
  \begin{tabular}{|RURR|EAMMM|}
    \hline
    \multicolumn{4}{|c|}{Opération} & \multicolumn{5}{c|}{Effet} \\
    \hline
    add24 & & r2, & o & & & r_2 + o & \mathrm{mod} & 24 \\
    add60 & & r2, & o & & & r_2 + o & \mathrm{mod} & 60 \\
    \hline
    mod   & & r2, & o & & & r_2     & \mathrm{mod} & o  \\
    mod24 & & r2  &   & & & r_2     & \mathrm{mod} & 24 \\
    mod60 & & r2  &   & & & r_2     & \mathrm{mod} & 60 \\
    \hline
  \end{tabular}
  \caption{Opérations arithmético-logiques}
\end{table}


\subsection{Opérations de manipulations de données}

Parmi les opérations du \mips, on conserve les suivantes. On a choisi de ne pas inclure d'opération de type \code{load\_immediate}, car on stockera dans certains registres les constantes nécessaires avant le lancement du processeur (comme 0, 24, 60\ldots).

\begin{table}[H]
  \centering
  \begin{tabular}{|RURR|EcM|}
    \hline
    \multicolumn{4}{|c|}{Opération} & \multicolumn{3}{c|}{Effet} \\
    \hline
    move & & r2 &      & & \affects & r_2 \\
    \hline
    lw   & & o  & (r2) & & \affects & \mathrm{RAM}[r_2 + o] \\
    rw   & & o  & (r2) & & $\rightarrow$ & \mathrm{RAM}[r_2 + o] \\
    \hline
  \end{tabular}
  \caption{Opérations de manipulations de données}
\end{table}


\subsection{Opérations de contrôle de l’exécution}

Enfin, on dispose des opérations suivantes afin de simuler des conditions et des boucles.

\begin{table}[H]
  \centering
  \begin{tabular}{|RRRR|McMM|}
    \hline
    \multicolumn{4}{|c|}{Opération} & \multicolumn{4}{c|}{Effet} \\
    \hline
    j   & o  &    &   & \mathrm{PC} & \affects & o & \\
    \hline
    beq & r, & o, & a & \mathrm{PC} & \affects & a & \text{si } $r = o$ \\
    \hline
    nop &    &    &   &             & & \varnothing & \\
    \hline
  \end{tabular}
  \caption{Opérations de contrôle de l’exécution}
\end{table}


\end{document}
